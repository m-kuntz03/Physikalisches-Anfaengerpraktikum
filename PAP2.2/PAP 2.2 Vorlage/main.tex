\documentclass{article}

\usepackage[T1]{fontenc}    %Schriftart des Dokumentes
\usepackage[ngerman]{babel} %Dokumentensprache, hier Deutsch
\usepackage{amsmath, amssymb, stmaryrd} %mathematische Schriftzeichen
\usepackage{graphicx} %Einfügen von Grafiken
\usepackage{wrapfig}
\usepackage{bm}
\usepackage{subfig}
\usepackage{newclude}
\usepackage{pdfpages}
\usepackage{hyperref}
\hypersetup{
    colorlinks,
    citecolor=black,
    filecolor=black,
    linkcolor=black,
    urlcolor=black
}

\makeatletter
\newcommand\invisiblesection[1]{%
  \refstepcounter{section}%
  \addcontentsline{toc}{section}{\protect\numberline{\thesection}#1}%
  \sectionmark{#1}\phantom{}}
\makeatother

\setlength{\parindent}{0pt} %Einrückung von Absätzen auf null gesetzt
\setlength{\parskip}{10pt} %Abstand zischen Absätzen auf 10pt gesetzt

\title{Versuch XXX: }
\author{Matthias Kuntz}
\date{00.06.2024}

\renewcommand*\contentsname{Zusammenfassung}

\begin{document}

\maketitle

\tableofcontents

\newpage

%-------------------------EINLEITUNG-------------------------
\section{Einleitung}

blabla


\subsection{Physikalische Grundlagen}


\newpage
\subsection{Versuchsaufbau}




\phantom{.}





%---------------VERSUCHSPROTOKOLL MIT MESSDATEN---------------
\newpage

\section{Versuchsprotokoll mit Messdaten}

\includegraphics[width=\textwidth]{graphics/mess1.jpg}
\newpage
\includegraphics[width=\textwidth]{graphics/mess2.jpg}
\newpage
\includegraphics[width=\textwidth]{graphics/mess3.jpg}
\newpage
\includegraphics[width=\textwidth]{graphics/mess4.jpg}
\newpage
\includegraphics[width=\textwidth]{graphics/mess5.jpg}
\newpage

\addtocounter{table}{6}




\clearpage
\newpage
%-------------------------AUSWERTUNG-------------------------
\section{Auswertung}

In dieser Evaluation werden alle Fehler, sofern keine spezifische Angabe gemacht wird, mithilfe der Gauss'schen Fehlerfortpflanzung berechnet. Dies bedeutet, dass ein Wert $F$, der mit der Formel $f(a_1, ..., a_n)$ berechnet wird, den Fehler $\Delta F$ annimmt:

\begin{equation}
    \Delta F = \sqrt{\sum_n \left( \frac{\partial f}{\partial a_n} \cdot \Delta a_n \right)^2}.
\end{equation}

Des Weiteren erfolgen Signifikanztests von zwei Werten $a$ und $a'$ über die folgende Formel:

\begin{equation}
    \sigma = \frac{|a-a'|}{\sqrt{(\Delta a)^2 + (\Delta a')^2}}.
\end{equation}

Die Auswertung sowie Berechnung erfolgen über das dem Dokument angehängte Python-Programm. Hierbei erfolgen Fits von Funktionen mithilfe der 'curve\_fit'-Funktion des 'SciPy'-Packages und Plots werden mit 'matplotlib' erstellt.

Die Güte eines Fits wird mit der $\chi^2$-Summe bewertet:

\begin{equation}
    \chi^2 = \sum_i^N \left( \frac{\textit{Funktionswert}_i - \textit{Messwert}_i}{\textit{Fehler}_i} \right)^2
\end{equation}

Auch verwendet wird $\chi^2_{red} = \chi^2 / f$, wobei der Freiheitsgrad $f$ die Anzahl der Messwerte minus die Anzahl der Fitparameter ist. Der auf die Freiheitsgrade normierte Wert soll bei einem guten Fit ungefähr 1 sein.


\newpage

\subsection{XXX}

\begin{table}[!h]
    \centering
    %\resizebox{\textwidth}{!}{
    \begin{tabular}{ccc}
        \hline
        $\bm{C}$ [nF] & $\bm{\omega}$ [kHz] & $\bm{L1}$ [H]  \\ \hline
         47 $\pm$ 5 & 24,19 $\pm$ 0,19 & 0,036 $\pm$ 0,004 \\
         47 $\pm$ 5 & 23,69 $\pm$ 0,19 & 0,038 $\pm$ 0,004 \\
         47 $\pm$ 5 & 23,88 $\pm$ 0,19 & 0,037 $\pm$ 0,004 \\ \hline
    \end{tabular}%}
    \caption{Berechnung der Induktivität L1}
    \label{tab:A4-L1}
\end{table}

\clearpage
\newpage
%---------------PRÄSENTATION DER ENDERGEBNISSE---------------
\section{Zusammenfassung der Endergebnisse}

bla


\newpage
%---------------ZUSAMMENFASSUNG UND DISKUSSION---------------
\section{Diskussion}

bla 

 
\newpage
%\includepdf[pagecommand=\invisiblesection{Python-Code},scale=0.8,pages=1]{241-Final.pdf}
%\includepdf[pagecommand={},scale=0.8,pages=2-last]{241-Final.pdf}

\end{document}

