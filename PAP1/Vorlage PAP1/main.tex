\documentclass{article}

\usepackage[T1]{fontenc}    %Schriftart des Dokumentes
\usepackage[ngerman]{babel} %Dokumentensprache, hier Deutsch
\usepackage{amsmath, amssymb, stmaryrd} %mathematische Schriftzeichen
\usepackage{graphicx} %Einfügen von Grafiken
\usepackage{wrapfig}
\usepackage{bm}

\setlength{\parindent}{0pt} %Einrückung von Absätzen auf null gesetzt
\setlength{\parskip}{10pt} %Abstand zischen Absätzen auf 10pt gesetzt

\title{Versuch X:}
\author{Matthias Kuntz}
\date{Das Datum}

\begin{document}

\maketitle

%-------------------------EINLEITUNG-------------------------
\section{Einleitung}

\subsection{Physikalische Grundlagen}

\subsection{Versuchsaufbau}

%---------------VERSUCHSPROTOKOLL MIT MESSDATEN---------------
\newpage

\section{Versuchsprotokoll mit Messdaten}

\includegraphics[width=\textwidth]{graphics/mess1.jpg}
\newpage
\includegraphics[width=\textwidth]{graphics/mess2.jpg}
\newpage
\includegraphics[width=\textwidth]{graphics/mess3.jpg}
\newpage

\addtocounter{table}{3}

%-------------------------AUSWERTUNG-------------------------
\section{Auswertung}

In dieser Evaluation werden alle Fehler, sofern keine spezifische Angabe gemacht wird, mithilfe der Gauss'schen Fehlerfortpflanzung berechnet. Dies bedeutet, dass ein Wert $F$, der mit der Formel $f(a_1, ..., a_n)$ berechnet wird, den Fehler $\Delta F$ gegeben über folgende Formel hat:

\begin{equation}
    \Delta F = \sqrt{\sum_n \left( \frac{\partial f}{\partial a_n} \cdot \Delta a_n \right)^2}.
\end{equation}

\subsection{X}

%---------------PRÄSENTATION DER ENDERGEBNISSE---------------
\section{Präsentation der Endergebnisse}

blabla.

%---------------ZUSAMMENFASSUNG UND DISKUSSION---------------
\section{Zusammenfassung und Diskussion}

blabla.

\end{document}

